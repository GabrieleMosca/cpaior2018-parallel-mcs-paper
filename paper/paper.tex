% vim: set spell spelllang=en tw=100 :

\documentclass{llncs}

\usepackage{amsmath}
\usepackage{amssymb}
\usepackage{cleveref}
\usepackage{tikz}
\usepackage{nicefrac}

% \usepackage{showframe}

\usetikzlibrary{decorations, decorations.pathreplacing, calc, backgrounds, positioning}

% cref style
\crefname{figure}{Fig.}{Figs.}
\Crefname{figure}{Fig.}{Figs.}

% pretty colours
\definecolor{uofgsandstone}{rgb}{0.321569, 0.278431, 0.231373}
\definecolor{magmapurple}{rgb}{0.505882, 0.145098, 0.505882}
\definecolor{magmaorange}{rgb}{0.984314, 0.529412, 0.380392}

\newcommand*\samethanks[1][\value{footnote}]{\footnotemark[#1]}

\title{Observations from Parallelising Three Maximum Common (Connected) Subgraph Algorithms}

\author{
    Ruth Hoffmann\inst{1}
    \and Ciaran McCreesh\thanks{This work was supported by the Engineering and Physical Sciences
        Research Council [grant numbers EP/K503058/1, EP/M508056, and EP/P026842/1]}\inst{2}
    \and Samba Ndojh Ndiaye\thanks{This work
    was supported by the ANR project SoLStiCe (ANR-13-BS02-0002-01)}\inst{3}
    \and Patrick Prosser\samethanks[1]\inst{2}
    \and Craig Reilly\samethanks[1]\inst{2}
    \and Christine Solnon\samethanks[2]\inst{4}
    \and James Trimble\samethanks[1]\inst{2}}

\institute{
    University of St Andrews, St Andrews, United Kingdom \and
    University of Glasgow, Glasgow, Scotland \and
    Universit\'e Lyon 1, LIRIS, UMR5205, F-69621, France
    \and INSA-Lyon, LIRIS, UMR5205, F-69621, France}

\begin{document}

\maketitle

\begin{abstract}
    We discuss our experiences adapting three recent algorithms for maximum common (connected)
    subgraph problems to exploit multi-core parallelism. These algorithms do not easily lend
    themselves to parallel search, as the search trees are extremely irregular, making balanced work
    distribution hard, and runtimes are very sensitive to value-ordering heuristic behaviour.
    Nonetheless, our results show that each algorithm can be parallelised successfully, with the
    threaded algorithms we create being clearly better than the sequential ones. We then look in
    more detail at the results, and discuss how speedups should be measured for this kind of
    algorithm.  Because of the difficulty in quantifying an average speedup when so-called anomalous
    speedups (superlinear and sublinear) are common, we propose a new measure called \emph{aggregate
    speedup}.
\end{abstract}

\section{Introduction}

Finding a maximum common subgraph is the key step in measuring the similarity or difference between
two graphs \cite{DBLP:journals/prl/Bunke97,DBLP:journals/prl/FernandezV01,o:Kriege15}.  Because of
this, maximum common subgraph problems frequently arise in biology and chemistry
\cite{DBLP:journals/jcamd/RaymondW02a,o:EhrlichR11,DBLP:journals/dam/GayFMSS14} where graphs
represent molecules or reactions, and also in computer vision
\cite{DBLP:journals/jair/CookH94,DBLP:conf/gbrpr/CombierDS13}, computer-aided manufacturing
\cite{o:LuoWSN17}, the analysis of programs and malware
\cite{DBLP:conf/icics/GaoRS08,DBLP:journals/compsec/ParkRS13}, crisis management
\cite{o:DelavalladeFLL16}, and social network analysis \cite{DBLP:journals/tkde/FangYZZ15}.

A \emph{subgraph isomorphism} is an injective mapping from a \emph{pattern} graph to a \emph{target}
graph which \emph{preserves adjacency}---that is, it maps adjacent vertices to adjacent vertices.
The isomorphism is \emph{induced} if additionally it maps non-adjacent vertices to non-adjacent
vertices, preserving non-adjacency as well. When working with labelled graphs, a subgraph
isomorphism must preserve labels, and on directed graphs, it must preserve orientation. A
\emph{common induced subgraph} of two graphs $G$ and $H$ is a pair of induced subgraph isomorphisms
from a pattern graph $P$, one to $G$ and one to $H$. A \emph{maximum common induced subgraph} is one
with as many vertices as possible. (The \emph{maximum common partial subgraph} problem is
non-induced, with as many edges as possible; this paper discusses only induced problems.) A common
variant of the problem requires a largest \emph{connected} subgraph
\cite{DBLP:journals/jcamd/RaymondW02a,DBLP:conf/mco/VismaraV08,o:EhrlichR11,o:LuoWSN17}.

Although both the connected and non-connected variants are NP-hard, recently progress has been made
towards solving the problem in practice.  This paper looks at three branch and bound algorithms for
maximum common (connected) induced subgraph problems, each of which is the state of the art for
certain classes of instance. We discuss our experiences in adding parallel tree-search to these
three algorithms. In each case, our results show that the parallel version of the algorithm is
clearly better than the sequential version, although a closer look at the results shows many
nuances. Thus this paper focusses primarily on presenting and interpreting the experimental data,
rather than heavy implementation details, in the hopes that the lessons we learned are helpful to
other practitioners---in particular, we introduce a new measure called \emph{aggregate speedup}
which is suitable for determining speedups for decision problems or optimisation problems where
anomalous speedups are common.

\section{Sequential Algorithms}

There are three competitive approaches for the maximum common subgraph problem, each being the
strongest on certain classes of instance. The first involves a reduction to the maximum clique
problem, whilst the other two approaches are inspired by constraint programming.

\subsection{Reduction to Maximum Clique}

A \emph{clique} in a graph is a subgraph where every vertex is adjacent to every other. There is a
well-known reduction from the maximum common subgraph problem to the problem of finding a maximum
clique in an \emph{association graph}
\cite{o:Levi73,DBLP:journals/jcamd/RaymondW02a,DBLP:conf/cp/McCreeshNPS16}; this reduction
resembles the microstructure encoding \cite{DBLP:conf/aaai/Jegou93a} of the constraint programming
approach described below. When combined with a modern maximum clique solver
\cite{DBLP:journals/ol/SegundoMRH13}, this is the current best approach for solving the problem on
labelled graphs \cite{DBLP:conf/cp/McCreeshNPS16}. A modified clique-like algorithm can also be
used to solve the maximum common connected subgraph problem, by ensuring connectedness during search
\cite{DBLP:conf/cp/McCreeshNPS16}; again, this is the best known way of solving the problem on
labelled graphs. However, the association graph encoding is extremely memory-intensive, limiting its
practical use to pairs of graphs with no more than a few hundred vertices.

\subsection{Constraint Programming}

The maximum common induced subgraph problem may be reformulated as a constraint optimisation
problem, as follows.  Observe that an equivalent definition of a common subgraph of graphs $G$ and
$H$ is an injective \emph{partial} mapping from $G$ to $H$ which preserves both adjacency and
non-adjacency.  Hence we pick whichever input graph has fewer vertices, and call it the
\emph{pattern}; the other graph is called the \emph{target}. The model then follows from this new
definition: for each vertex in the pattern, we create a variable, whose domain ranges over each
vertex in the target graph, plus an additional value $\bot$ representing an unmapped vertex. We then
have three sets of constraints.  The first set says that for each pair of adjacent vertices in the
pattern (that is, for each edge in the pattern), if neither of these vertices are mapped to $\bot$
then these vertices must be mapped to an adjacent pair of target vertices. The second set is
similar, but looks at non-adjacent pairs (or non-edges).  Finally, the third set ensures
injectivity, by enforcing that the variables must be all different except when using $\bot$. This
final set of constraints may either be implemented using binary constraints between all pairs of
variables, or a special global ``all different except $\bot$'' propagator
\cite{DBLP:conf/cp/PetitRB01}. The objective is simply to find an assignment of values to
variables, maximising the number of variables not set to $\bot$. The state of the art for this
technique is a dedicated (non-toolkit) implementation of a forward-checking branch and bound search
over this model \cite{DBLP:conf/cp/NdiayeS11,DBLP:conf/cp/McCreeshNPS16}.

Two approaches exist for ensuring connectedness: either a conventional global constraint and
propagator can be used \cite{DBLP:conf/cp/McCreeshNPS16}, or a special branching rule can
enforce connectedness during search \cite{DBLP:conf/mco/VismaraV08}. The two techniques are
broadly comparable performance-wise \cite{DBLP:conf/cp/McCreeshNPS16}, but the branching rule is
simpler to implement.

\subsection{Domain Splitting (McSplit and McSplit$\downarrow$)}

McCreesh et al.\ \cite{o:McCreeshPT17} observe that due to the special structure of the maximum
common subgraph problem, the following property holds throughout the search process using the
constraint programming model: any two variables either have domains with no values in common (with
the possible exception of $\bot$), or have identical domains. The McSplit algorithm exploits this
property. It explores essentially the same search tree as the basic forward-checking constraint
programming approach, but using different supporting algorithms and data structures.  Rather than
storing a domain for each vertex in the pattern graph, equivalence classes of vertices in both
graphs are stored in a special data structure which is modified in-place and restored upon
backtracking. This enables fast propagation of the constraints and smaller memory requirements. In
addition, this data structure enables stronger branching heuristics to be calculated cheaply. The
McSplit algorithm effectively dominates conventional constraint programming approaches, being
consistently over an order of magnitude faster.

The McSplit$\downarrow$ algorithm is a variant designed for instances where we expect nearly all of
the smaller graph to be found. It branches first on result size, from largest possible result
downwards.

\subsection{$k$-less Subgraph Isomorphism}

A different take on the constraint programming approach is presented by
Hoffmann et al.\ \cite{DBLP:conf/aaai/HoffmannMR17}. They approach maximum common subgraph via the
subgraph isomorphism problem, asking the question ``if a pattern graph cannot be found in the
target, how much of the pattern graph can be found?''. The $k{\downarrow}$ algorithm tries to solve
the subgraph isomorphism problem first for $k=0$ (asking whether the whole pattern graph can be found in the
target). Should that not be satisfiable, it tries to solve the problem for $k=1$ (one vertex cannot
be matched), and should that also not be satisfiable, it iteratively increases $k$ until the result
is satisfiable. This approach exploits strong invariants using paths and the degrees of vertices to
prune large portions of the search space.

This algorithm is aimed primarily at large instances, where the two graphs are of different orders,
and where it is expected that the solution will involve most of the smaller graph (that is, $k$ is
expected to be low). The sequential implementation we start with does not support labels or the
connected variant.

\section{Benchmark Instances}

Most of the benchmark instances we will use come from a standard database for maximum common
subgraph problems \cite{DBLP:journals/prl/SantoFSV03,DBLP:journals/jgaa/ConteFV07}. This benchmark
set can be used in a number of ways, for different variants of the problem. Following other recent
work \cite{DBLP:conf/cp/McCreeshNPS16,DBLP:conf/aaai/HoffmannMR17,o:McCreeshPT17}, we use it to
create five families of instances, as follows:

\begin{description}
    \item[Unlabelled] undirected instances, by selecting the first ten members of each parameter
        class where the graphs have up to 50 vertices each---this gives us a total of 4,110
        instances.

    \item[Vertex labelled] undirected instances, by selecting the first ten members of each
        parameter class (and so graphs have up to 100 vertices each), using the 33\%
        labelling scheme \cite{DBLP:journals/prl/SantoFSV03} for vertices only. This gives 8,140 instances.

    \item[Both labelled, directed] instances, by selecting the first ten members of each parameter
        class, and applying the 33\% labelling scheme \cite{DBLP:journals/prl/SantoFSV03} to both
        vertices and edges. Again, this gives 8,140 instances.

    \item[Unlabelled, connected] instances, as per the \emph{unlabelled} case.

    \item[Both labelled, connected] instances, starting in the same way as the \emph{both labelled,
        directed} case. These are then converted to undirected graphs by treating edges as
        undirected, picking the label of the lower-numbered edge.
\end{description}

\noindent
Following Hoffmann et al.\ \cite{DBLP:conf/aaai/HoffmannMR17}, we also work with the 5,725 \textbf{Large} instances
originally introduced for studying portfolios of subgraph isomorphism algorithms
\cite{DBLP:conf/lion/KotthoffMS16}. These graphs are unlabelled and undirected, and can include up
to 6,671 vertices. We do not use the clique encoding on these instances due to its
memory requirements.

\section{Parallel Search}

The clique and $k{\downarrow}$ algorithms already make use of fine-granularity bit-parallelism. To
introduce coarse-grained thread parallelism, we will parallelise search: viewing backtracking search
as forming a tree, we can explore different portions of the tree using different threads. We use a
shared incumbent, so better solutions found by one thread can be used by others immediately. In this
paper we use C++11 native threads, and so only support shared memory systems.

Parallel tree-search has a long history \cite{o:BaderHC05}. Of particular interest to us are
so-called \emph{anomalies}
\cite{DBLP:journals/cacm/LaiS84,DBLP:journals/tc/LiW86,DBLP:conf/irregular/BruinKT95}: because we
are not performing a fixed amount of work, we should have no expectation of a linear speedup, and
instead we could see a sublinear speedup (much less than $n$ from $n$ processors, if speculative
work turns out to be wasted) or a superlinear speedup (much more than $n$ from $n$ processors, if a
strong incumbent is found more quickly). An absolute slowdown (a speedup much less than 1) is also
possible when using some parallelisation techniques.

We stress that these anomalies are due to changes in the amount of work done, and are not due to
work balance problems (although work balance is \emph{also} unusually difficult for this problem).
Anomalies can have a very strong effect on these algorithms, and we will therefore try to mitigate
them as far as possible. In the evaluation of their ``embarrassingly parallel search'' technique,
Malapert et al.\ \cite{DBLP:journals/jair/MalapertRR16} ``consider unsatisfiable, enumeration and
optimization [problem] instances'', and ``ignore the problem of finding a first feasible solution
because the parallel speedup can be completely uncorrelated to the number of workers, making the
results hard to analyze''. They do ``consider optimization problems for which the same variability
can be observed, but at a lesser extent because the optimality proof is required''. Unfortunately,
many of the instances we consider behave more like decision problem instances than optimisation
instances: due to the combination of a low solution density, good value-ordering heuristics, and a
strong bound function in cases where the optimal solution is relatively large, it is often the case
that the runtime is determined almost entirely by how long it takes to find an optimal solution,
with the proof of optimality being nearly trivial.  Indeed, attempts to parallelise the basic
constraint programming approach by static decomposition have had limited success
\cite{DBLP:conf/ictai/MinotNS15}.

\subsection{Parallel Maximum Clique}

Thread-parallel versions of state-of-the-art maximum clique algorithms already exist.
McCreesh et al.\ \cite{DBLP:journals/topc/McCreeshP15} compare several of these approaches, and make
an important observation: although work balance is a problem due to the irregularity of the search
tree, often the interaction between search order and parallel work decomposition is the dominating
factor in determining speedups. They explain why anomalies are in fact common in practice: many
clique problem instances benefit immensely from having found a strong incumbent, but have solutions
which are either unique or rare, and are hard to find. They propose a work splitting mechanism which
offsets anomalies, guaranteeing reproducibility (two runs with the same instance on the same
hardware will give similar runtimes), scalability (increasing the number of cores cannot make things
worse), and no absolute slowdowns.  Additionally, this mechanism explicitly offsets the commitment
to early branching choices, where search ordering heuristics are most likely to be inaccurate
\cite{DBLP:conf/ijcai/HarveyG95,DBLP:conf/cp/ChuSS09}, making superlinear speedups common.

We will use this mechanism for our experiments.  The clique-based maximum common subgraph algorithm
effectively differs only in the preprocessing stage, and the clique-inspired connected algorithm
described by McCreesh et al.\ \cite{DBLP:conf/cp/McCreeshNPS16} is sufficiently similar that it may be parallelised in
exactly the same way. Based upon preliminary experiments, we set the mechanism's splitting depth
limit parameter to be five rather than the original three, since maximum common subgraph instances
appear to give even more irregular search trees than normal clique problem instances.

\subsection{Parallel Constraint-Based Search}

A similar approach may be used for the $k{\downarrow}$ algorithm. Although it is not quite a
conventional branch and bound algorithm, each individual $k$ pass is a tree-search, and may be
parallelised. For each pass, we use the same work splitting mechanism as in the clique algorithm,
starting by splitting only at the top level of search to explicitly introduce diversity, and then
iteratively increasing the splitting depth as additional work is needed (up to a limit of five
levels deep).  Because the $k{\downarrow}$ algorithm uses a conventional constraint programming
domain store, there is no need to use recomputation; the state is naturally copied at each branching
point.

In principle the McSplit algorithm may be parallelised in exactly the same way. However, this
algorithm makes heavy use of an in-place, backtrackable data structure, which is not copied for
recursive calls. In order to introduce the \emph{potential} for parallelism, we must make
copies of the state data structure. Implemented na{\"\i}vely, this can give an order of magnitude
slowdown to the sequential algorithm, which can be hard to recover using parallelism. To lessen the
effects, rather than copying state for each recursive call, we copy once before the main branching
loop, and then copy that copy in each ``helper'' thread, replaying the branching loop without making
duplicate recursive calls.  (We believe a better approach using partial recomputation may be
possible, and intend to investigate this further in the future.)

\section{Empirical Evaluation}

We perform our experiments on systems with dual Intel Xeon E5-2697A v4 processors and 512GBytes RAM,
running Ubuntu 17.04, with GCC 6.3.0 as the compiler. Each machine has a total of thirty-two cores.
We run all our experiments with a one thousand second timeout for each instance. All of our
sequential runtimes are from optimised implementations by their original authors which were not
designed with parallelism in mind---that is, speedups from parallelism are genuine improvements over
the state of the art.

\subsection{Parallel Search is Better Overall}

\begin{figure}[p]
    \includegraphics*{gen-graph-cumulative-plain.pdf}
    \hfill
    \includegraphics*{gen-graph-cumulative-33v.pdf}

    \vspace*{1em}

    \includegraphics*{gen-graph-cumulative-33ved.pdf}
    \hfill
    \includegraphics*{gen-graph-cumulative-plain-connected.pdf}

    \vspace*{1em}

    \includegraphics*{gen-graph-cumulative-33ve-connected.pdf}
    \hfill
    \includegraphics*{gen-graph-cumulative-sip.pdf}

    \caption{The cumulative number of instances solved over time. Except in the bottom left plot,
    the 32 threaded parallel versions (shown using dotted lines) are always better in aggregate than
    the sequential versions (shown using solid lines).}\label{figure:cumulative}
\end{figure}

In \cref{figure:cumulative} we plot empirical cumulative distribution functions showing the number
of instances solved over time, for both sequential (solid lines) and parallel (dotted lines)
versions of each algorithm. To read these plots, make a choice of timeout along the $x$-axis (which
uses a log scale). The $y$ value at that point shows the number of instances whose runtime
(individually) is at most $x$, for a particular algorithm. In other words, at any given $x$ value,
the highest line shows which algorithm is able to solve the largest number of instances using a
per-instance timeout of that $x$ value, bearing in mind that the actual sets of instances solved by
each algorithm may be completely different.

With one exception, each plot gives the same conclusion: if we are working with a solving time of at
least $100$ milliseconds, then for any problem family and any sequential algorithm, if given the
option of switching to the corresponding parallel algorithm, then we should do so. For the McSplit
algorithm on both labelled, connected instances, the parallel algorithm does not quite catch up to
the sequential algorithm.

Although good at showing general trends, cumulative plots can hide interesting details. We therefore
now take a closer look at each of the three algorithms in turn.

\subsection{Clique Results In Depth}

\begin{figure}[p]
    \centering
    \begin{tikzpicture}%{{{
        \node[anchor=west] (C1) {\includegraphics*{gen-graph-scatter-plain-clique-vs-clique-par-t32.pdf}};
        \node[anchor=west, right = 0.1 of C1.east] (C2) {\includegraphics*{gen-graph-histogram-plain-clique-vs-clique-par-t32.pdf}};
        \node[anchor=west, right = 0.1 of C2.east] (C3) {\includegraphics*{gen-graph-pp2-plain-clique-vs-clique-par-t32.pdf}};
        \node[anchor=west, left = 0.1 of C1.east] (C0) at (0, 0) {\rotatebox{90}{\scriptsize Unlabelled}};
    \end{tikzpicture}

    \centering
    \begin{tikzpicture}%{{{
        \node[anchor=west] (C1) {\includegraphics*{gen-graph-scatter-33v-clique-vs-clique-par-t32.pdf}};
        \node[anchor=west, right = 0.1 of C1.east] (C2) {\includegraphics*{gen-graph-histogram-33v-clique-vs-clique-par-t32.pdf}};
        \node[anchor=west, right = 0.1 of C2.east] (C3) {\includegraphics*{gen-graph-pp2-33v-clique-vs-clique-par-t32.pdf}};
        \node[anchor=west, left = 0.1 of C1.east] (C0) at (0, 0) {\rotatebox{90}{\scriptsize Vertex labelled}};
    \end{tikzpicture}

    \centering
    \begin{tikzpicture}%{{{
        \node[anchor=west] (C1) {\includegraphics*{gen-graph-scatter-33ved-clique-vs-clique-par-t32.pdf}};
        \node[anchor=west, right = 0.1 of C1.east] (C2) {\includegraphics*{gen-graph-histogram-33ved-clique-vs-clique-par-t32.pdf}};
        \node[anchor=west, right = 0.1 of C2.east] (C3) {\includegraphics*{gen-graph-pp2-33ved-clique-vs-clique-par-t32.pdf}};
        \node[anchor=west, left = 0.1 of C1.east] (C0) at (0, 0) {\rotatebox{90}{\scriptsize Both labelled,}};
        \node[anchor=west, right = -0.2 of C0.east] (C0) {\rotatebox{90}{\scriptsize directed}};
    \end{tikzpicture}

    \centering
    \begin{tikzpicture}%{{{
        \node[anchor=west] (C1) {\includegraphics*{gen-graph-scatter-plain-connected-clique-vs-clique-par-t32.pdf}};
        \node[anchor=west, right = 0.1 of C1.east] (C2) {\includegraphics*{gen-graph-histogram-plain-connected-clique-vs-clique-par-t32.pdf}};
        \node[anchor=west, right = 0.1 of C2.east] (C3) {\includegraphics*{gen-graph-pp2-plain-connected-clique-vs-clique-par-t32.pdf}};
        \node[anchor=west, left = 0.1 of C1.east] (C0) at (0, 0) {\rotatebox{90}{\scriptsize Unlabelled,}};
        \node[anchor=west, right = -0.2 of C0.east] (C0) {\rotatebox{90}{\scriptsize connected}};
    \end{tikzpicture}

    \centering
    \begin{tikzpicture}%{{{
        \node[anchor=west] (C1) {\includegraphics*{gen-graph-scatter-33ve-connected-clique-vs-clique-par-t32.pdf}};
        \node[anchor=west, right = 0.1 of C1.east] (C2) {\includegraphics*{gen-graph-histogram-33ve-connected-clique-vs-clique-par-t32.pdf}};
        \node[anchor=west, right = 0.1 of C2.east] (C3) {\includegraphics*{gen-graph-pp2-33ve-connected-clique-vs-clique-par-t32.pdf}};
        \node[anchor=west, left = 0.1 of C1.east] (C0) at (0, 0) {\rotatebox{90}{\scriptsize Vertex labelled,}};
        \node[anchor=west, right = -0.2 of C0.east] (C0) {\rotatebox{90}{\scriptsize connected}};
    \end{tikzpicture}

    \caption{In the left column, per-instance speedups, using the clique algorithm. The $x$-axis is
    sequential performance and the $y$-axis is 32 threaded performance. In the centre, histograms
    plotting the distribution of speedups for instances whose sequential runtime was at least 500
    milliseconds, and below the timeout. On the right, performance
    profiles.}\label{figure:cliquespeedups}
\end{figure}

\begin{figure}[tb]
    \centering
    \includegraphics*{gen-graph-as-clique.pdf}

    \caption{Aggregate speedups from 32 threads, shown as a function of sequential runtime, for each
    family supported by the clique algorithm.}\label{figure:cliqueas}
\end{figure}

In the first column of \cref{figure:cliquespeedups}, we see scatter plots comparing the sequential and
parallel runtimes of the clique algorithm on an instance by instance basis, using a log-log plot.
Each point represents one instance, with the $x$-axis being the sequential runtime and the $y$-axis
the parallel runtime.  Instances which timed out using one algorithm but not the other are shown as
points along the outer borders. Points below the $x{-}y$ diagonal line represent speedups. The
colour of the points indicates the relative size of the solution---darker points represent instances
where the solution uses most of the vertices of the input graphs. (We use these conventions for
scatter plots throughout this paper.)

Broadly speaking, the results are similar on each of the five families. For runtimes below 100
milliseconds, overheads and the preprocessing step dominate, and we are usually only able to achieve
a small speedup. At higher runtimes, most speedups appear to be between ten and thirty, except on
the final family of both labelled connected instances, where they are mostly between five and ten.
For a few instances, the speedups are lower (but they are still clearly speedups), whilst in the
first four families, we also see evidence of superlinear speedups being relatively common.

However, attempting to determine a speedup by staring at a scatter plot is not particularly
quantitative. We \emph{could} attempt to find a best fit line through these points, pretending that
the superlinear speedups are outliners. We might perhaps get away with this if outliers were rare
enough, but in practice we are not expecting linear speedups (and for the other two algorithms, we
will see that superlinear speedups are even more common). Alternatively, we could rig our
experiments to remove anomalies, by priming search with a known-optimal solution; however, since the
time to find an optimal solution (but not prove its optimality) is so important, we do not consider
this to be a fair measure of algorithm performance \cite{DBLP:journals/topc/McCreeshP15}.

A more principled approach is given in the second column of \cref{figure:cliquespeedups}. For
instances where the sequential run both succeeded and took at least 500 milliseconds, we plot the
distribution of speedups obtained. These histograms confirm our informal observations. However,
these plots are still not especially satisfactory: in order to calculate a speedup, we can only
consider instances where the sequential algorithm succeeded, and so these plots underestimate
superlinear speedups. The choice of a 500 millisecond minimum sequential runtime is also rather
arbitrary, and is acceptable only if we expect the parallel algorithms will only be used on
relatively hard instances.

In the third column we show performance profiles \cite{DBLP:journals/mp/DolanM02}. A performance
profile is a cumulative plot of how many times worse the performance of an algorithm is relative to
the virtual best algorithm. Each plot shows three options as different lines. The `all' lines
include easy instances whose sequential runtime is below 500 milliseconds, whilst the other two
lines exclude them. The `hard' line treats sequential timeouts as having been solved at the time
limit, whilst the `PAR10' line treats timeouts as taking ten times longer than the timeout (this
convention is common in portfolios \cite{DBLP:conf/aaai/XuHL10}). The solid lines show the
sequential algorithms, whilst the dotted lines show the parallel algorithms. (There are no dotted
lines on the top four plots for the `hard' and `PAR10' cases, since the parallel algorithm always
beats the sequential algorithm in these cases.) We have normalised the $y$-axis to the number of
counted instances in a given class.

Unfortunately, these three lines can paint very different pictures. For example, for unlabelled
instances on the top row, if we include easy instances, it appears that the parallel algorithm can
be up to ten times worse, whereas if we exclude them, it is never worse. If we do not use the PAR10
scheme, the performance profile also suggests that there are around twenty-five percent of the
hard instances where the speedup is below 10, whilst using PAR10 correctly shows that such instances
are rare. However, PAR10 is only effective in this regard because the ``typical'' speedup is in the
region of 10 (and this is a particular inconvenience because we seek a way of characterising
speedups which does not rely upon us already knowing that 10 is a reasonable choice of penalty).

A further problem is that to deal with the large superlinear speedups sometimes observed, a log
scale must be used on the $x$-axis; this makes speedups of 10 and 30 look very similar, whilst in
practice the difference is important.

To avoid these weaknesses, we propose a new way of characterising speedups. Refer back to the
cumulative plots in \cref{figure:cumulative}. The usual way of comparing two algorithms on these
plots is by measuring the vertical difference between lines, which would tell us how many more
instances the parallel algorithm can solve than the sequential algorithm can with a particular
choice of timeout.  However, measuring the \emph{horizontal} distance between lines also conveys
information. Suppose the sequential algorithm can solve $y$ instances with a selected timeout of
$s$. By moving to the left on a cumulative plot, we can find the timeout $p$ required for the
parallel algorithm to solve the same number of instances, bearing in mind that \emph{the two sets of
instances could have completely different members}. We define the \emph{aggregate speedup} to be
$\nicefrac{s}{p}$; this can be expressed as a function of time (i.e.\ $s$) or of the number of
instances solved ($y$).

We plot aggregate speedups as a function of time in \cref{figure:cliqueas}. For a sequential timeout
of one thousand seconds, we get speedups of thirty to forty in the unlabelled, vertex labelled, and
both labelled, directed cases. In the unlabelled cases, our aggregate speedup are over thirty-two,
which is superlinear.  With some detailed knowledge of the underlying sequential algorithm, this
should perhaps not surprise us: for instances with a large solution, once we have found that
solution, a proof of optimality is relatively easy. However, finding that solution can be unusually
hard, particularly since the branching strategy for the connected constraint necessarily interferes
with the tailored search order used by modern clique algorithms.  In contrast, for the both labelled
connected case, our aggregate speedup is barely larger than one. A closer inspection of the results
shows that the search tree is unusually narrow and deep for these instances, making work balance
harder and contention higher.

What about scalability and reproducibility? The first plot in \cref{figure:cliquescale} shows the
effects of going from sequential to threaded with two cores, and the next four plots show the
effects of doubling the number of threads each time. These plots show that most of the superlinear
effects occur with fairly small numbers of threads, with nearly all of the benefits of increased
diversity in search being obtained once eight threads are used. As expected, in no case does
increasing the number of threads make things substantially worse.
The final plot in \cref{figure:cliquescale} shows that runtimes are reproducible: running the same
instance on the same hardware twice takes almost exactly the same amount of time.

\begin{figure}[p]
    \includegraphics*{gen-graph-scatter-33ved-clique-vs-clique-par-t2.pdf}
    \hfill
    \includegraphics*{gen-graph-scatter-33ved-clique-par-t2-vs-clique-par-t4.pdf}
    \hfill
    \includegraphics*{gen-graph-scatter-33ved-clique-par-t4-vs-clique-par-t8.pdf}

    \vspace*{0.2em}

    \includegraphics*{gen-graph-scatter-33ved-clique-par-t8-vs-clique-par-t16.pdf}
    \hfill
    \includegraphics*{gen-graph-scatter-33ved-clique-par-t16-vs-clique-par-t32.pdf}
    \hfill
    \includegraphics*{gen-graph-scatter-33ved-clique-par-t32r-vs-clique-par-t32.pdf}

    \caption{Per-instance speedups from the clique algorithm on vertex- and edge-labelled, directed
    instances, when going from sequential to two threads in the first plot, then increasing the
    number of threads in subsequent plots. The final plot shows 32 threads versus a repeated run
    also with 32 threads.}\label{figure:cliquescale}
\end{figure}

\begin{figure}[p]
    \includegraphics*{gen-graph-scatter-plain-kdown-vs-kdown-par-t32.pdf}
    \hfill
    \includegraphics*{gen-graph-scatter-sip-kdown-vs-kdown-par-t32.pdf}
    \hfill
    \includegraphics*{gen-graph-scatter-plain-kdown-par-t4-vs-kdown-par-t16.pdf}

    \vspace*{0.2em}

    \includegraphics*{gen-graph-scatter-plain-kdown-par-t32r-vs-kdown-par-t32.pdf}
    \hfill
    \includegraphics*{gen-graph-as-kdown.pdf}

    \caption{In the first two plots, per-instance speedups, using the $k{\downarrow}$ algorithm. The
    $x$-axis is sequential performance and the $y$-axis is 32 threaded performance. Next,
    scalability and reproducibility, and finally, aggregate speedups for both
    families.}\label{figure:kdownscatters}
\end{figure}

These results are comforting: they show that anomalies can be controlled, and that
switching to a parallel algorithm is not only better, but also safe from a scientific
reproducibility perspective.

\subsection{$k{\downarrow}$ Results In Depth}

In \cref{figure:kdownscatters} we show per-instance and aggregate speedups for the $k{\downarrow}$
algorithm. On unlabelled instances, we see a range of speedups between 0.9 and ten, with an
aggregate speedup of seven. These results are not as good as with the clique algorithm.  Profiling
suggests memory allocation problems: although the amount of work done would suggest good
parallelism, the time taken to perform each domain copy operation increases as the number of threads
increases. Unlike the clique algorithm, which has very small, cache-friendly data structures which
are modified in-place, the state for the $k{\downarrow}$ algorithm is large and much of the runtime
is spent copying data structures. (Our hardware is a dual multi-core processor configuration, and
each core has its own low-level cache, but memory bandwidth is shared. Interestingly, on older Xeon
E5 v2 systems, this problem is much more pronounced.)

For the large instances, our aggregate speedup is higher, at around twenty. This has two causes: for
larger graphs, the computational effort per recursive call increases by more than the amount the
memory copying does, reducing the memory problem slightly, and additionally a much larger
number of superlinear speedups occurred with this family of instances. We could perhaps anticipate
this latter effect: in many of these instances the maximum common subgraph covers all or nearly all
of the smaller of the two graphs, and so once it is found, the proof of optimality is trivial.
However, finding a witness can be difficult. We should also expect value-ordering heuristics in
these algorithms to be weak at the top of search (they are based upon degree, and many graphs do not
have a large degree spread), and so the benefits of high-up diversity can be extremely large
\cite{DBLP:conf/ijcai/HarveyG95,DBLP:conf/cp/ChuSS09,DBLP:journals/topc/McCreeshP15}. Indeed,
similar results were seen with a parallel version of the subgraph isomorphism algorithm upon which
$k{\downarrow}$ is based \cite{DBLP:conf/cp/McCreeshP15}.

The third and fourth plots in \Cref{figure:kdownscatters} show that as with the clique algorithm,
this parallelism is reproducible, and that runtimes do not get worse when the number of threads is
increased. (Although not shown, we also tried to parallelise $k{\downarrow}$ using randomised
work-stealing from Intel Cilk Plus. Doing so gives generally reasonable results on average, as it
does for the clique algorithm \cite{DBLP:journals/topc/McCreeshP15}, but now repeat runtimes can
differ by more than an order of magnitude.)

\subsection{McSplit Results In Depth}

\begin{figure}[p]
    \includegraphics*{gen-graph-scatter-plain-mcsplit-vs-mcsplit-par-t32.pdf}
    \hfill
    \includegraphics*{gen-graph-scatter-33v-mcsplit-vs-mcsplit-par-t32.pdf}
    \hfill
    \includegraphics*{gen-graph-scatter-33ved-mcsplit-vs-mcsplit-par-t32.pdf}

    \vspace*{0.2em}

    \includegraphics*{gen-graph-scatter-plain-connected-mcsplit-vs-mcsplit-par-t32.pdf}
    \hfill
    \includegraphics*{gen-graph-scatter-33ve-connected-mcsplit-vs-mcsplit-par-t32.pdf}
    \hfill
    \includegraphics*{gen-graph-scatter-sip-mcsplit-vs-mcsplit-par-t32.pdf}

    \vspace*{2em}

    \begin{minipage}[c]{0.62\textwidth}
    \includegraphics*{gen-graph-as-mcsplit.pdf}
    \end{minipage}
    \hfill
    \begin{minipage}[t]{0.35\textwidth}
    \includegraphics*{gen-graph-scatter-plain-mcsplit-par-t4-vs-mcsplit-par-t16.pdf}
    \vspace*{0.2em}
    \includegraphics*{gen-graph-scatter-plain-mcsplit-par-t32r-vs-mcsplit-par-t32.pdf}
    \end{minipage}

    \caption{On the first two rows, per-instance speedups, using McSplit. Below, aggregate speedups
    on the left, and on the right, scalability and reproducibility.}\label{figure:mcsplitscatter}
\end{figure}

Finally, we look at our attempts to parallelise the McSplit algorithm. Recall that doing so required
heavy modifications to the implementation, introducing significant amounts of speculative copying of a
data structure that is usually backtrackable and modified in-place.

For unlabelled, unlabelled connected, and large instances, \cref{figure:mcsplitscatter} shows a
particularly high proportion of strongly superlinear speedups. This is because the McSplit algorithm
is focussed upon exploring the search space very quickly, and its branching heuristics do not have
the advantage of the domain filtering performed by $k{\downarrow}$, or the rich inter-domain
knowledge coming from the combination of the association graph encoding and the colour ordering used
by clique algorithms.  Thus making a correct value-ordering choice at the top of search is harder
for McSplit than for other algorithms, and so increased diversity can be particularly beneficial.

For the large instances, we see evidence of work balance problems. McSplit's use of a ``smallest
domain first'' variable-ordering heuristic, combined with the presence of $\bot$ in domains, tends
to produce narrow (nearly binary) and deep search trees. These balance problems are even more
evident in the labelled cases (where following a guessed assignment, many domains are left with only
two values), and often lead to little to no speedup being obtained. Indeed, for the labelled,
connected case, we see a slight aggregate \emph{slowdown}.

The scatter plots also show occasional large absolute slowdowns, sometimes by over an order of
magnitude.  These are due to the changes which had to be made to the sequential algorithm (and
because we are benchmarking against the sequential algorithm, not a parallel algorithm with one
thread), rather than search order effects. In cases where parallelism cannot be exploited, the cost
of speculatively copying domains at each level of search can dominate the runtimes. Because of this,
fixing work balance problems by increasing the splitting depth typically makes matters much worse,
not better.

What about scalability and reproducibility? \Cref{figure:mcsplitscatter} presents a
less ideal picture than for the previous two algorithms---again, this is due to speculative
overheads that fail to pay off, rather than being anomalies in the classical sense.

\section{Conclusion}

We have parallelised three state-of-the-art maximum common (connected) subgraph algorithms with a
reasonable degree of success by using dynamic work-splitting. Despite having a branch and bound
flavour, all three sequential algorithms had their own difficulties and performance characteristics
which prevented them from cleanly fitting into common abstraction frameworks. Nonetheless, our
results show that the parallel algorithms are not just better in aggregate, but also preserve the
desirable reproducibility properties of sequential algorithms. A large part of our success was down
to using parallelism to explicitly introduce diversity into the search process, offsetting weak
early value-ordering branching choices.

There is room for improvement, particularly with respect to work balance. However, improvements to
work balance must not come at the expense of the search order properties, nor at the cost of
increased overheads.

More generally, we introduced the idea of \emph{aggregate speedups}, to deal with measuring a
speedup in the presence of anomalies. This measure gives sensible answers even when working with
instances which behave like decision problems. Aggregate speedups informed part of our analysis, but
our results highlight the importance of viewing results in multiple ways, and in using large
families of instances with different characteristics when evaluating parallel search
algorithms---had we looked only at unlabelled instances, or only at labelled connected instances,
our conclusion would be very different.

\bibliographystyle{splncs}
\bibliography{combined}

\end{document}
